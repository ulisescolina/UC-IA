\documentclass{article}
\renewcommand\refname{Referencias}
\renewcommand\contentsname{\'Indice de Conten\'ido}
\usepackage{graphicx}
\graphicspath{{IMG/}}
\usepackage{caption}
\usepackage{subcaption}
\usepackage{float}

\title{\textsc{Inteligencia Artificial\\Articulo Cient\'ifico}}
\author{Ulises C. Ramirez}
\date{13 de Septiembre, 2018}

\begin{document}
\maketitle
\pagenumbering{gobble}
\newpage
\section*{Versionado}
Para el corriente documento se est\'a llevando un versionado a fin de mantener un respaldo del trabajo y adem\'as proveer a la c\'atedra o a cualquier interesado la posibilidad de leer el material en la \'ultima versi\'on disponible.\\

\begin{center}
  \textsc{Repositorio}: \textit{https://github.com/ulisescolina/UC-IA}
\end{center}

\hfill--\textsc{Ulises}
\newpage
% === Inicio del Cuerpo del Documento === %
\pagenumbering{arabic}
\textsc{Consigna}: \textit{\textbf{Los invito a buscar un articulo científico sobre temas vinculados a la inteligencia artificial y colocarlos en el Foro describiendo que les llamo la atención de este articulo y porque lo seleccionaron algunas propuestas estan en: $Documentos \rightarrow Clase1 \rightarrow Material primer clase$}}


% === Bilbiografia === %

\begin{thebibliography}{99}
	% Item 1
	\bibitem[Mohamadyari, et al, 2017]{mohamadyari2017} \textsc{Mohamadyari, Saina; Attar, Niousha; Aliakbary, Sadegh}. \textit{On Feature prediction in Temporal Social Networks based on Artificial Neural Network Learning}. Ferdowsi University of Mashhad. 7th International Conference on Computer and Knowledge Engineering, ICCKE, 2017.
	% Item 2
	\bibitem[]{}
\end{thebibliography}
\end{document}
