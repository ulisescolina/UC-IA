\documentclass{article}
\renewcommand\refname{Referencias}
\renewcommand\contentsname{\'Indice de Conten\'ido}
\usepackage{graphicx}
\graphicspath{{IMG/}}
\usepackage{caption}
\usepackage{subcaption}
\usepackage{float}
\usepackage{booktabs}
\usepackage{amsmath}


\title{\textsc{Inteligencia Artificial y Sistemas Expertos}\\Probabilidad
Condicionada}
\author{Ulises C. Ramirez}
\date{8 de Octubre, 2018}
\begin{document}
\maketitle
\pagenumbering{gobble}
\newpage
\section*{Versionado}
Para el corriente documento se est\'a llevando un versionado a fin de mantener un respaldo del trabajo y adem\'as proveer a la c\'atedra o a cualquier interesado la posibilidad de leer el material en la \'ultima versi\'on disponible.\\

\begin{center}
  \textsc{Repositorio}: \textit{https://github.com/ulisescolina/UC-IA/}
\end{center}

\hfill--\textsc{Ulises}
\pagenumbering{gobble}
\newpage

% === Inicio del Cuerpo del Documento === %
\pagenumbering{arabic}
\section*{Probabilidad Condicionada}
En este documento se provee un ejemplo de la aplicaci\'on de lo visto en
clase a un ejemplo inventado por el alumno tomando como base el ejemplo
prove\'ido por la c\'atedra \cite{sebpia2018}.

\subsection*{Definici\'on del escenario}
Se ha recibido una embarcacion con contenedores que poseen placas madre para
ordenadores, en total la cantidad de \textit{items} que se tienen es de
18800, luego de un an\'alisis del inventario se encontr\'o que los
items pertenecian a las marcas \texttt{M1}, \texttt{M2}, \texttt{M3} y
\texttt{M4}. Otra de las cuestiones que fueron de inter\'es en el an\'alisis
del cargamento es determinar con que \textit{features} dentro del abanico
de car\'acteristicas de placas madre contaba cada placa, se encontraron las
siguientes \texttt{C1}, \texttt{C2} y \texttt{C3}.

Se debe poder determinar lo siguiente:
\begin{enumerate}
\item Probabilidad a priori de cada \texttt{Marca} si se presenta un item al
azar
\item La probabilidad que se presente una \texttt{Marca} sabiendo que el
\texttt{item} tiene todos los \textit{features}.
\end{enumerate}

Se tiene que la cantidad de placas seg\'un cada fabricante es la siguiente:
\begin{center}
$Total=18800$\\
$M1 = 3500$\\
$M2 = 5500$\\
$M3 = 2500$\\
$M4 = 7300$\\
\end{center}

\begin{table}[ht]
\caption{Datos tabulados para el contenido recibido}
\centering
\begin{tabular}{c | c c c }
\hline
		& \texttt{C1} & \texttt{C2} & \texttt{C3} \\ \hline
\texttt{M1} 	&     3130   &      502    &     1244     \\
\texttt{M2} 	&     2325   &      5351    &     124     \\
\texttt{M3} 	&     273   &      2312    &     14     \\
\texttt{M4} 	&     135   &      7281    &     4214  \\ \hline
\end{tabular}
\label{tab:datoscontenido}
\end{table}

\subsection*{Probabilidades para $M_{i}$}
Para realizar \'estos c\'alculos se realizar\'a la operaci\'on presentada en
la ecuaci\'on \ref{eq1}

\begin{equation} \label{eq1}
	P(M_{i}) = \frac{\#M_{i}}{Total}
\end{equation}

A continuaci\'on, en los \texttt{Desarrollos \ref{des1}, \ref{des2}, \ref{des3}} y
\texttt{\ref{des4}} se exponen los diferentes resultados obtenidos del c\'alculo
de probabilidades a priori para $M_{i}$

\begin{equation} \label{des1}
\begin{split}
 P(M_{1}) & =  \frac{3500}{18800} \\
 & = 0,186170
\end{split}
\end{equation}

\begin{equation} \label{des2}
\begin{split}
 P(M_{2}) & =  \frac{5500}{18800} \\
 & = 0,292553
\end{split}
\end{equation}

\begin{equation} \label{des3}
\begin{split}
 P(M_{3}) & =  \frac{2500}{18800} \\
 & = 0,132978
\end{split}
\end{equation}

\begin{equation} \label{des4}
\begin{split}
 P(M_{4}) & =  \frac{7300}{18800} \\
 & = 0,388297
\end{split}
\end{equation}

\subsection*{Probabilidades para \texttt{$C_{i}$}}
En esta secci\'on se presenta el c\'alculo realizado para poder definir las
diferentes probabilidades a priori de encontrar la caracter\'istica $C_{i}$ en el total
de placas madre y lo adjuntamos a la \texttt{Tabla \ref{tab:datoscontenido}},
estas probabilidades est\'an adheridas a la \texttt{Tabla \ref{tab:probci}}.

El c\'alculo para las probabilidades a priori para las  caracter\'isticas las
podemos calcular seg\'un \cite{sebpia2018}, como presenta la ecuaci\'on
\ref{eq6}

\begin{equation} \label{eq6}
P(C_{i})=\sum_{j=1}^{N}P(M_{i})P(C_{i}|M_{j})
\end{equation}

Siendo $N$ la cantidad de, en este caso, \texttt{marcas} se tenga en el
inventario, para nuestro caso particular se tiene que $N=4$

Otra forma de verlo, es la siguiente

\begin{equation} \label{eq7}
P(C_{i})=\frac{\sum_{j=1}^{N} \#C_{i}\text{ en }M_{j}}{\#Total}
\end{equation}

conceptualmente, lo que expresa la ecuaci\'on \ref{eq7} es que se deben sumar
los cardinales para cada columna y este resultado dividirlo por el total de
items.

Los calculos para las probabilidades de $C_{i}$ se encuentran en los
\texttt{Desarrollos \ref{des5}, \ref{des6}} y \texttt{\ref{des7}}

\begin{equation} \label{des5}
\begin{split}
 P(C_{1}) & =  \frac{3130 + 2325 + 273 + 135}{18800} \\
 & = 0,311861
\end{split}
\end{equation}

\begin{equation} \label{des6}
\begin{split}
 P(C_{2}) & =  \frac{502 + 5351 + 2312 + 7281}{18800} \\
 & = 0,821595
\end{split}
\end{equation}

\begin{equation} \label{des7}
\begin{split}
 P(C_{3}) & =  \frac{1244 + 124 + 14 + 4214}{18800} \\
 & = 0,297659
\end{split}
\end{equation}

\begin{table}[ht]
\caption{Probabilidades para $C_{i}$}
\centering
\begin{tabular}{c | c c c }
\hline
		& \texttt{C1} & \texttt{C2} & \texttt{C3} \\ \hline
\texttt{M1} 	&     3130   &      502    &     1244     \\
\texttt{M2} 	&     2325   &      5351    &     124     \\
\texttt{M3} 	&     273   &      2312    &     14     \\
\texttt{M4} 	&     135   &      7281    &     4214  \\ \hline
Probabilidad	& 0,311861	& 0,821595	& 0,297659	\\
\hline
\end{tabular}
\label{tab:probci}
\end{table}

\subsection*{C\'alculo de verosimilitudes}
Luego de este c\'alculo se tendr\'a una nueva tabla la cual representar\'a la
verosimilitud de cada caracter\'istica $C_{j}$ dada una marca $M_{i}$, para
realizar el c\'alculo se procede a realizar lo siguiente

\begin{equation} \label{eq8}
P(C_{j}|M_{i})=\frac{\#C_{j}}{\#M_{i}}
\end{equation}

Esto es, la cardinalidad de un $C_{j}$ sobre la cardinalidad de un $M_{i}$. A
continuaci\'on se presenta la tabla nueva de la que se habl\'o.

\begin{table}[ht]
\caption{Verosimilitudes}
\centering
\begin{tabular}{c | c c c }
\hline
		& \texttt{C1} & \texttt{C2} & \texttt{C3} \\ \hline
\texttt{M1} 	&     0,894285   &   0,143428    &     0,355428     \\
\texttt{M2} 	&     0,422727   &   0,972909    &     0,022545     \\
\texttt{M3} 	&     0,109200   &   0,924800    &     0,005600     \\
\texttt{M4} 	&     0,018493   &   0,997397    &     0,577260     \\ \hline
\end{tabular}
\label{tab:verosimilitudes}
\end{table}

\subsection*{C\'alculo de probabilidades condicionales}
Este se llevar\'a a cabo mediante la siguiente ecuaci\'on:

\begin{equation} \label{eq8}
P(M_{i}|C_{1}, C_{2}, \cdots,
C_{j})=P(M_{i})\prod_{j=1}^{N}P(C_{j}|M_{i})
\end{equation}

La ecuaci\'on \ref{eq8} se puede interpretar conceptualmente como, la
probabilidad de que ocurra que se seleccione la marca $M_{i}$ dado a que esta
marca cuente con los features $C_{1}$, $C_{2}$,\cdots,$C_{j}$.

\begin{equation} \label{des8}
\begin{split}
P(M_{1}|C_{1}, C_{2}, C_{3}) & =
0,186170\cdot0,894285\cdot0,143428\cdot0,355428 \\
& = 0,008487
\end{split}
\end{equation}

\begin{equation} \label{des9}
\begin{split}
P(M_{2}|C_{1}, C_{2}, C_{3}) & =
0,292553\cdot0,422727\cdot0,972909\cdot0,022545 \\
& = 0,002712
\end{split}
\end{equation}

\begin{equation} \label{des10}
\begin{split}
P(M_{3}|C_{1}, C_{2}, C_{3}) & =
0,132978\cdot0,109200\cdot0,924800\cdot0,005600 \\
& = 0,000075
\end{split}
\end{equation}

\begin{equation} \label{des11}
\begin{split}
P(M_{4}|C_{1}, C_{2}, C_{3}) & =
0,388297\cdot0,018443\cdot0,997397\cdot0,577260 \\
& = 0,004123
\end{split}
\end{equation}

Ahora si realizamos la sumatoria de todos los resultados obtenidos
anteriormente, tendremos que, $\sum_{i=1}^{N=4}P(M_{i}|C_{1, 2, 3})=0,015397$
esto es un indicio de que la normalizaci\'on es necesaria en el ejercicio, para
esto, procedemos a dividir a cada resultado de $P(M_{i}|C_{1, 2, 3})$ por la
suma obtenida anteriormente.

Finalizamos entonces con lo siguiente:

\begin{equation} \label{des12}
\frac{P(M_{1}|C_{1}, C_{2}, C_{3})}{0,015397} & =  0,551211\\
\end{equation}

\begin{equation} \label{des13}
\frac{P(M_{2}|C_{1}, C_{2}, C_{3})}{0,015397} & =  0,176138\\
\end{equation}

\begin{equation} \label{des14}
\frac{P(M_{3}|C_{1}, C_{2}, C_{3})}{0,015397} & =  0,004871\\
\end{equation}

\begin{equation} \label{des15}
\frac{P(M_{4}|C_{1}, C_{2}, C_{3})}{0,015397} & =  0,267779\\
\end{equation}

La sumatoria de estos nuevos valores normalizados es de $0,999999$, lo cual
basicamente tiende a 1, es decir, el 100\%, el resultado que se aprecia es
consecuencia de lo conocido como \textit{error de truncamiento}.

% === Bilbiografia === %
\newpage
\begin{thebibliography}{99}
	% Item 1
	\bibitem[IA y SE - 2018]{sebpia2018}\textsc{Inteligencia Artificial y
Sistemas Expertos}.
\textit{Sistemas Basados en Probabilidades - Ejemplo. Clase 7}.
\end{thebibliography}
\end{document}
