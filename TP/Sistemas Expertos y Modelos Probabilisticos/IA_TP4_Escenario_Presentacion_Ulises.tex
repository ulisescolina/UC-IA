%----------------------------------------------------------------------------------------
%	Paquetes y Temas/Colores
%----------------------------------------------------------------------------------------

\documentclass{beamer}

\mode<presentation> {
\usetheme{Madrid}
%\setbeamertemplate{footline} % Remover el pie de pagina
%\setbeamertemplate{footline}[page number] % Reemplazar el pie de pagina con un numero de pagina
\setbeamertemplate{navigation symbols}{} % remover simbolos de navegacion
}

\usepackage{graphicx} % Permite la insercion de imagenes
\usepackage{booktabs} % Permite la manipulacion de tablas
\usepackage{amsmath} % Permite la manipulacion de ecuaciones

%----------------------------------------------------------------------------------------
%	Caratula
%----------------------------------------------------------------------------------------

\title[IAySE - 2018]{\textsc{Inteligencia Artificial y Sistemas
Expertos}\\Probabilidad Condicionada\\--Ejemplo--} % Primero aparece en el pie de pagina (titulo corto),
		   % el segundo aparece solo en la caratula (titulo largo)

\author{\textsc{Ulises C. Ramirez}
}
\institute[UNaM -- FCEQyN] % Institucion
{
\texttt{\{ulir19, ulisescolrez\}@gmail.com}\\
\medskip
Universidad Nacional de Misiones\\Facultad de Ciencias Exactas, Qu\'imicas y
Naturales \\ % Institucion para caratula
}
\date{8 de Octubre, 2018}

\begin{document}

\begin{frame}
\titlepage % Caratula
\end{frame}

\begin{frame}
\frametitle{Vista Preliminar} % Vistazo general de los temas
\tableofcontents % Contenido, section{} y subsection{}
\end{frame}

%----------------------------------------------------------------------------------------
%	Diapositivas
%----------------------------------------------------------------------------------------

%------------------------------------------------
\section{Presentaci\'on del Problema}
%------------------------------------------------
\begin{frame}
\frametitle{Inventario Placas Madre}
Se ha recibido una embarcaci\'on con contenedores que poseen placas madre para
ordenadores, en total la cantidad de items que se tienen es de 18800, luego de
un análisis del inventario se encontró que los items pertenecían a las marcas
M1, M2, M3 y M4. Otra de las cuestiones que fueron de interés en el análisis
del cargamento fue determinar con que características dentro del abanico de carácteristicas
de placas madre contaba cada item, se encontraron las siguientes
características C1, C2 y C3.
\end{frame}
%------------------------------------------------
\begin{frame}
\frametitle{Inventario Placas Madre}
Se ha recibido una embarcaci\'on con contenedores que poseen placas madre para
ordenadores, en total la cantidad de items que se tienen es de \textbf{18800},
luego de un análisis del inventario se encontró que los items pertenecían a las \textbf{marcas
M1, M2, M3} y \textbf{M4}. Otra de las cuestiones que fueron de interés en el análisis
del cargamento fue determinar con que \textbf{características} dentro del abanico de carácteristicas
de placas madre contaba cada item, se encontraron las características
\textbf{C1, C2} y \textbf{C3}.
\end{frame}
%------------------------------------------------
\begin{frame}
\frametitle{Datos de inter\'es}
\end{frame}
%------------------------------------------------
\begin{frame}
\frametitle{Datos de inter\'es}
\begin{block}{Marcas}
$M_{1}, M_{2}, M_{3}, M_{4}$
\end{block}

\end{frame}
%------------------------------------------------
\begin{frame}
\frametitle{Datos de inter\'es}
\begin{block}{Marcas}
$M_{1}, M_{2}, M_{3}, M_{4}$
\end{block}

\begin{block}{Características}
$C_{1}, C_{2}, C_{3}$
\end{block}
\end{frame}

%------------------------------------------------

\begin{frame}
\frametitle{Datos del Problema}
\end{frame}

%------------------------------------------------

\begin{frame}
\frametitle{Datos del problema}
\begin{center}
\textsc{-- Cardinalidad de Marcas --}\\
$Total=18800$ \\
$M1=3500$	;	$M2=5500$\\
$M3=2500$	;	$M4=7300$
\end{center}
\begin{table}[ht]
\caption{Datos tabulados para el contenido recibido}
\centering
\begin{tabular}{c | c c c }
\hline
		& \texttt{C1} & \texttt{C2} & \texttt{C3} \\ \hline
\texttt{M1} 	&     3130   &      502    &     1244     \\
\texttt{M2} 	&     2325   &      5351    &     124     \\
\texttt{M3} 	&     273   &      2312    &     14     \\
\texttt{M4} 	&     135   &      7281    &     4214  \\ \hline
\end{tabular}
\label{tab:datoscontenido}
\end{table}
\end{frame}

%------------------------------------------------

\begin{frame}
\frametitle{Consignas}
\begin{itemize}
\item Encontrar la probabilidad de cada marca si se presenta un item al azar.
\end{itemize}
\end{frame}

%------------------------------------------------

\begin{frame}
\frametitle{Consignas}
\begin{enumerate}
\item Encontrar la probabilidad de cada marca si se presenta un item al azar.
\item La probabilidad de que se presente una marca sabiendo que el item tiene
todas las caracter\'isticas.
\end{enumerate}
\end{frame}

%------------------------------------------------
\section{Soluci\'on propuesta}
%------------------------------------------------
\begin{frame}
\frametitle{Solucion a la consigna 1}
\begin{itemize}
\item La soluci\'on a esta consigna es sencilla, simplemente se solicita la
probabilidad a priori de cada marca en el mont\'on de items.
\end{itemize}
\end{frame}

%------------------------------------------------
\begin{frame}
\frametitle{Solucion a la consigna 1}
\begin{itemize}
\item La soluci\'on a esta consigna es sencilla, simplemente se solicita la
probabilidad a priori de cada marca en el mont\'on de items.
\item Para esto solo se necesita conocer lo siguiente
\end{itemize}

\begin{equation*} \label{eq1}
	P(M_{i}) = \frac{\#M_{i}}{Total}
\end{equation*}
\end{frame}

%------------------------------------------------

\begin{frame}
\frametitle{Solucion a la consigna 1}
\begin{itemize}
\item La soluci\'on a esta consigna es sencilla, simplemente se solicita la
probabilidad a priori de cada marca en el mont\'on de items.
\item Para esto solo se necesita conocer lo siguiente
\end{itemize}

\begin{equation*} \label{eq1}
	P(M_{i}) = \frac{\#M_{i}}{Total}
\end{equation*}

De esta forma se obtienen los valores para:\\
$P(M_{1})=0,186170$, $P(M_{2})=0,292553$, $P(M_{3})=0,132978$,
$P(M_{4})=0,388297$
\end{frame}

%------------------------------------------------
\begin{frame}
\frametitle{Solucion a la consigna 2}
\begin{itemize}
\item Consignas estan relacionadas.
\item Los datos obtenidos volveran a ser utilizados.
\end{itemize}
\end{frame}

%------------------------------------------------

\begin{frame}
\frametitle{Solucion a la consigna 2}
Adelant\'andonos, lo que se va a necesitar hacer es lo siguiente:
\begin{equation*} \label{eq8}
P(M_{i}|C_{1}, C_{2}, \cdots,
C_{j})=P(M_{i})\prod_{j=1}^{N}P(C_{j}|M_{i})
\end{equation*}
\end{frame}

%------------------------------------------------

\begin{frame}
\frametitle{Solucion a la consigna 2}
Adelant\'andonos, lo que se va a necesitar hacer es lo siguiente:
\begin{equation*} \label{eq8}
P(M_{i}|C_{1}, C_{2}, \cdots,
C_{j})=\textcolor{green}{P(M_{i})}\textcolor{orange}{\prod_{j=1}^{N}P(C_{j}|M_{i})}
\end{equation*}
Para el calculo de la parte naranja, procedemos a encontrar las
verosimilitudes.
\end{frame}

%------------------------------------------------

\begin{frame}
\frametitle{Resultado de \prod_{j=1}^{N}P(C_{j}|M_{i})}
Teniendo en cuenta que $P(C_{j}|M_{i})$ se calcula de la siguiente manera
\begin{equation*} \label{eq8}
P(C_{j}|M_{i}) = \frac{\#C_{j}}{\#M_{i}}
\end{equation*}
\end{frame}

%------------------------------------------------

\begin{frame}
\frametitle{Resultado de \prod_{j=1}^{N}P(C_{j}|M_{i})}
Teniendo en cuenta que $P(C_{j}|M_{i})$ se calcula de la siguiente manera
\begin{equation*} \label{eq8}
P(C_{j}|M_{i}) = \frac{\#C_{j}}{\#M_{i}}
\end{equation*}
\begin{table}[ht]
\caption{Verosimilitudes}
\centering
\begin{tabular}{c | c c c }
\hline
		& \texttt{C1} & \texttt{C2} & \texttt{C3} \\ \hline
\texttt{M1} 	&     0,894285   &   0,143428    &     0,355428     \\
\texttt{M2} 	&     0,422727   &   0,972909    &     0,022545     \\
\texttt{M3} 	&     0,109200   &   0,924800    &     0,005600     \\
\texttt{M4} 	&     0,018493   &   0,997397    &     0,577260     \\ \hline
\end{tabular}
\label{tab:verosimilitudes}
\end{table}
\end{frame}

%------------------------------------------------

\begin{frame}
\frametitle{Soluciones para $P(M_{i}|C_{j})$}
\begin{equation*}
P(M_{1}|C_{1}, C_{2}, C_{3}) = 0,008487
\end{equation*}

\begin{equation*}
P(M_{2}|C_{1}, C_{2}, C_{3}) = 0,002712
\end{equation*}

\begin{equation*}
P(M_{3}|C_{1}, C_{2}, C_{3}) = 0,000075
\end{equation*}

\begin{equation*}
P(M_{4}|C_{1}, C_{2}, C_{3})  = 0,004123
\end{equation*}

Si ahora realizamos:

\begin{equation*}
\sum_{i=1}^{N=4}P(M_{i}|C_{1, 2, 3}) = 0,015397
\end{equation*}

\end{frame}

%------------------------------------------------

\begin{frame}
\frametitle{Normalizaci\'on}
Para esto realizamos lo siguiente:
\begin{equation*}
\frac{P(M_{i}|C_{1}, C_{2}, C_{3})}{\sum_{i=1}^{N=4}P(M_{i}|C_{1, 2,
3})}
\end{equation*}

\begin{center}
\textsc{--------- Ahora todo normalizado ---------}
\end{center}
\begin{equation*}
P(M_{1}|C_{1}, C_{2}, C_{3}) = 0,551211
\end{equation*}

\begin{equation*}
P(M_{2}|C_{1}, C_{2}, C_{3}) = 0,176138
\end{equation*}

\begin{equation*}
P(M_{3}|C_{1}, C_{2}, C_{3}) = 0,004871
\end{equation*}

\begin{equation*}
P(M_{4}|C_{1}, C_{2}, C_{3}) = 0,267779
\end{equation*}
\end{frame}

%------------------------------------------------
\begin{frame}
\frametitle{Referencias}
\footnotesize{
	\begin{thebibliography}{99}
		% Item 1
		\bibitem[<++>]{<++>} \textsc{<++>}. \textit{<++>}
		\newblock <++>
	\end{thebibliography}
}
\end{frame}

%----------------------------------------------------------------------------------------

\end{document}
