\documentclass{article}
\renewcommand\refname{Referencias}
\renewcommand\contentsname{\'Indice de Conten\'ido}
\usepackage{graphicx}
\graphicspath{{IMG/}}
\usepackage{caption}
\usepackage{subcaption}
\usepackage{float}

\title{\textsc{Inteligencia Artificial\\Sistemas Expertos y Modelos de Redes
Probabil\'isticos}}
\author{Ulises C. Ramirez}
\date{01 de Octubre, 2018}
\begin{document}
\maketitle
\pagenumbering{gobble}
\newpage
\section*{Versionado}
Para el corriente documento se est\'a llevando un versionado a fin de mantener
un respaldo del trabajo y adem\'as proveer a la c\'atedra o a cualquier interesado
la posibilidad de leer el material en la \'ultima versi\'on disponible.\\

\begin{center}
  \textsc{Repositorio}: \textit{https://github.com/ulisescolina/UC-IA}
\end{center}

\hfill--\textsc{Ulises}

\tableofcontents
\pagenumbering{gobble}
\newpage

% === Inicio del Cuerpo del Documento === %
\pagenumbering{arabic}
\section{¿Qu\'e es un sistema experto?}
\label{sec:sysex}
Seg\'un \cite{castillo} y adem\'as la presentaci\'on de la c\'atedra
\cite{iase2018}, los cuales citan un estudio publicado por [Stevens, 1984]
un Sistema Experto, \textit{es una entidad que tiene la
facultad de pensar y razonar como lo haria alguien con conocimiento en un
\'area en concreto}, adem\'as se puede agregar que este debe, continuamente
estar relacion\'andose con los datos que conoce y realizando
modificaciones en los mismos a fin de producir resultados con mas significado
para los que reciban la informaci\'on. Las mismas fuente provee un ejemplo que es el de un Sistema
Experto para diagn\'ostico medico, este podr\'ia tomar datos relevantes del
caso, realizar una comparacion con el conocimiento que tiene alojado en la base
de datos y dar un diagn\'ostico.

\cite{castillo} adem\'as ofrece su propia definici\'on de qu\'e es un sistema
experto: \textit{Un Sistem Experto puede definirse como un sistema
inform\'atico (hardware y software) que simula a los expertos humanos en un
\'area de especializaci\'on dada}

Se pueden hablar de caracter\'isticas generales de los Sistemas Expertos, estas
se mencionan a continuaci\'on:
\begin{itemize}
\item Procesr y memorizar informaci\'on.
\item Razonar ante situaciones, ya sea en las cuales este t\'enga experiencia o
que a\'un no se le hayan presentado.
\item Comunicaci\'on con el ser humano y/u otros sistemas expertos.
\end{itemize}


\section{¿Qu\'e tipos de sistemas expertos existen}
\label{sec:sysext}
En la misma fuente que habla acerca de las cuestiones tratadas en la
\texttt{Secci\'on \ref{sec:sysex}}, se presenta una clasificaci\'on de sistemas
expertos, se asume que esta clasificaci\'on es a lo que se hace referencia
cuando se habla de ``tipos de sistemas expertos''.\\

Clasificaci\'on de Sistemas Expertos:
\begin{itemize}
\item Sistemas Expertos basados en Reglas.
\item Sistemas Expertos basados en Probabilidades o Est\'ocasticos.
\item Sistemas Expertos Difusos.
\end{itemize}

Aunque, como menciona \cite{castillo}, un sistema experto se puede clasificar
en dos tipos principales seg\'un la naturaleza de problemas para los que
est\'an diseñados: deterministas y estoc\'asticos.\\

Se decide, optar por lo segundo, e ir con lo que menciona Castillo, y decir que
los tipos de sistemas expertos existentes son \textit{deterministas} y
\textit{estoc\'asticos}.


\section{Componentes y tareas}
\label{sec:sysextcomponentesytareas}
\textsc{Consigna}: \textbf{¿Cu\'ales son los componentes de un sistema experto
y que tareas realiza cada uno de ellos?}\\

Los componentes de un sistema experto seg\'un se detalla en \cite{castillo} se
explican a continuaci\'on.

\subsection{La Componente Humana}
\label{sub:comphum}
Hace referencia a la relacion existente entre \textit{expertos humanos
especialistas en el tema de estudio} y los \textit{ingenieros del
conocimiento}. La tarea que realiza esta componente es el suministrar el
conocimiento b\'asico en el tema de interes y trasladarlo a un lenguaje que
pueda ser entendido por el sistema experto.

\subsection{La Base del Conocimiento}
\label{sub:basecon}
\'Esta es presentada a los ingenieros del conocimiento como una base ordenada y
estructurada, y un conjunto de relaciones bien definidas y explicadas. Esta
base busca la permanencia dentro de la memoria de un sistema experto, no ser
efimero como lo son los datos. Este conocimiento entonces se basa en
afirmaciones de validez general, tales como reglas, distribuciones de
probabilidad, etc.


\subsection{Subsistema de Adquisici\'on de Conocimiento}
\label{sub:adqcono}
Este controla el flujo del nuevo conocimiento que va desde el experto humano a
la base de datos. El sistema experto se encarga de determinar que nuevo
conocimiento se necesita, si lo que se recibe es nuevo, y se encarga de
incorporarlo.

\subsection{Control de la Coherencia}
Este controla la consistencia de la base de datos y evita que unidades de
conocimiento inconsistentes entren en la misma. Estas existen debido a que
existe la posibilidad de que incluso un experto humano pueda formular
afirmaciones inconsistentes con lo que ya existe, en caso de que no se cuente
con un control de coherencia esta unidad contradictoria formulada por el humano
puede llegar a formar parte de la base de datos. Adem\'as el sistema informa a
expertos humanos las restriciones que debe cumplir la informaci\'on que \'estos
provean para ser coherente on la base de conocimientos y asi asiste al ser
humano a dar informaci\'on fiable.

\subsection{El Motor de Inferencia}
\label{sub:motorinf}
La funci\'on principal de esta componente es el de obtener conclusiones
aplicando el conocimiento a los datos. Estas conclusiones se pueden basar en
\textit{conocimiento determinista} o \textit{conocimiento probabil\'istico}. Es
responsable tambien de la propagacion de conocimiento incierto, esta
propagacion de incertidumbre es uno de los eslabones m\'as d\'ebiles de un
sistema experto.

\subsection{El Subsistema de Adquisici\'on de Conocimiento}
\label{sub:adqcono2}
\begin{center}
**Esto parece ser un error de tipeo dentro del libro**\\
*** Repetido con la Secci\'on \ref{sub:adqcono} **
\end{center}

Este es utilizado por el motor de inferencia discutido en \texttt{Secci\'on
\ref{sub:motorinf}} cuando el conocimiento inicial es muy limitado y no se
pueden sacar conclusiones, el objetivo de esta relaci\'on es obtener conocimiento
necesario y continuar con el proceso de inferencia hasta llegar a las conclusiones.

\subsection{Interfase de Usuario}
Este es el enlace entre el sistema experto y el usuario. Su proposito es dotar
al sistema experto de mecanismos que permitan obtener y presentar informaci\'on
de manera f\'acil y agradable, haciendo que el sistema experto en cuesti\'on
sea una herramienta efectiva. Trabaja en relaci\'on cercana con lo presentado en
la \texttt{Secci\'on \ref{sub:adqcono2}} ya que al no poder llegar a una
inferencia por medio del Motor de Inferencia, este motor solicita la ayuda al
Subsistema de adquisicion de conocimiento, que luego podr\'ia solicitar al
usuario la informaci\'on necesaria, para realizar esta acci\'on es menester
contar con una interfaz que comunique al usuario con el sistema experto.


\subsection{El Subsistema de Ejecuci\'on de \'Ordenes}
Esta es la componente que permite al  sistema experto iniciar acciones, las
cuales se basan en las conclusiones a las que llega el Motor de Inferencia.

\subsection{El Subsistema de Explicaci\'on}
Este se encarga de presentar el proceso seguido por el motor de inferencia para
llegar a la conclusi\'on, en muchos dominios de aplicaciones esto es totalmente
necesario debido a los riesgos asociados con las acciones a ejecutar.

\subsection{El subsistema de Aprendizaje}
Dos caracter\'isticas  que resaltan de un sistema experto son:
\begin{itemize}
\item La capacidad de aprendizaje, y
\item La capacidad de obtener experiencia a partir de datos disponibles
\end{itemize}

En cuanto al apredizaje, se puede diferenciar dos tipos: \textit{aprendizaje
estructural} y \textit{aprendizaje par\'ametrico}.

\textbf{Aprendizaje Estructural}: esto hace referencia a aspectos relacionados
con la estructura del conocimiento (reglas, distribuciones de probabilidad,
etc), la inclusi\'on de una nueva regla a la base del conocimiento es un
ejemplo de aprendizaje estructural.\\

\textbf{Aprendizaje Param\'etrico}: se refiere a la estimaci\'on de parametros
necesarios para construir una base de conocimiento. Una estimaci\'on de
probabilidad asociada a un s\'intoma o enfermedad es ejemplo de aprendizaje
param\'etrico.\\

En cuanto a la obtenci\'on de experiencia, los datos pueden ser obtenidos por
expertos y no expertos y puede utilizarse por el subsistema de adquisici\'on de
conocimiento y por el subsistema de aprendizaje.

\section*{Recapitulaci\'on}
De las componentes mencionadas, puede verse que los sistemas expertos pueden
realizar varias tareas. Estas incluyen, pero no se limitan a, las
siguientes:
\begin{itemize}
\item Adquisici\'on del conocimiento y la verificaci\'on de su coherencia; por
lo que el sistema experto puede ayudar a los expertos humanos a dar
conocimiento coherente.
\item Memorizar conocimiento.
\item Aprender de la base del conocimiento y de los datos disponibles.
\item Realizar inferencia y razonamiento en situaciones deterministas y de
incertidumbre.
\item Explicar conclusiones o acciones tomadas.
\item Comunicar con los expertos y no expertos y con otros sistemas
expertos.
\end{itemize}


% === Bilbiografia === %
\newpage
\begin{thebibliography}{99}
	% Item 1
	\bibitem[Clase 5, IA - Sistemas Expertos ,
2018]{iase2018}\textsc{C\'atedra: Inteligencia Artificial y Sistemas Expertos}.
\textit{Presentacion: Clase5 - Sistemas Expertos}.
	% Item
	%\bibitem[Clase 6, IA - S]
	% Item
	\bibitem[Castillo, et al]{castillo}\textsc{Castillo, Enrique;
Guti\'errez, Jos\'e Manuel; Hadi, Ali S.}. \textit{Sistemas Expertos y Modelos
de Redes Probabil\'isticas}.
\end{thebibliography}
\end{document}
