\documentclass{article}
\renewcommand\refname{Referencias}
\renewcommand{\contentsname}{\'Indice de Conten\'ido}
\usepackage{graphicx}
\graphicspath{{img/}}
\usepackage{caption}
\usepackage{subcaption}
\usepackage{float}
\usepackage[spanish]{babel}
\usepackage[style=ieee]{biblatex}
\addbibresource{./IA_TP8_Ulises.bib}

\title{\textsc{Inteligencia Artificial y Sistemas Expertos}\\Trabajo Práctico
Número 8}
\author{\textsc{Ulises C. Ramirez} [\textit{ulir19@gmail.com}]}
\date{\today}
\begin{document}
\maketitle
\pagenumbering{gobble}
\newpage

\section*{Versionado}
Para el corriente documento se est\'a llevando un versionado a fin de mantener
un respaldo del trabajo y adem\'as proveer a la c\'atedra o a cualquier
interesado a la posibilidad de leer el material en la \'ultima versi\'on disponible.\\

\begin{center}
  \textsc{Repositorio}: \textit{https://github.com/ulisescolina/UC-IA}
\end{center}

\hfill--\textsc{Ulises}
\tableofcontents
\pagenumbering{gobble}
\newpage

% === Inicio del Cuerpo del Documento === %
\pagenumbering{arabic}
\section{Perceptron}
{\label{sec:perceptron}}

\subsection{Aprendizaje: Supervisado y No-Supervisado}
La principal caracteristicas entre estos tipos de aprendizaje es la dependencia
de cada uno para con un agente externo al algoritmo que determina si las
acciones tomadas por la entidad inteligente son correctas o no, en el
aprendizaje no-supervisado, no se tiene tal cosa, mientras en el supervisado
esto es necesario.

\subsection{Elegir dos modelos de red y clasificarlos}
\subsubsection{Back Propagation Network (BPN)}
\begin{itemize}
	\item Tipo de Aprendizaje: Supervisado, con correccion de error.
	\item Cantidad de Capas: Multicapa.
	\item Tipo de Conexiones: Feed-Forward.
\end{itemize}

\subsubsection{Red de Hopfield}
\begin{itemize}
	\item Tipo de Aprendizaje: No-Supervisado, Hebbiano.
	\item Cantidad de Capas: Monocapa.
	\item Tipo de Conexiones: Conexiones laterales con las neuronas del mismo
		nivel, no permite conexiones autorecurrentes.
\end{itemize}


\subsection{Ejercicio implementación AND}
Se opto por realizar la implementación del mismo mediante la codificación del
algoritmo, puntualmente se realizo la actividad con el lenguaje Python y el
lenguaje JavaScript, si bien se solicita la implementación de una compuerta AND
se hace entrega de un perceptron que se puede entrenar para aprender a
clasificar puntos dentro de un plano. El ejercicio puede ser encontrado en la
carpeta llamada \texttt{Perceptron}, alli encontrará dos subdirectorios,
llamados \texttt{Python} y \texttt{JS} en donde se tendrán las respectivas
implementaciones en cada lenguaje, asegurese de leer el \texttt{README.md} en
cada uno de estos directorios dado a que alli se encuentran algunas acotaciones
a tener en cuenta a la hora de hacer funcionar los algoritmos.

\section{Hopfield}
{\label{sec:hopfield}}

\subsection{Otras aplicaciones de Hopfield}
Una de las aplicaciones mas basicas de las redes de Hopfield con las que nos
podemos cruzar varias veces en un dia es la solución de Captchas en diferentes
paginas Web que evitan que bots sobrecargen, con proposito malicioso, de
peticiones a una pagina web, ademas su utilizacion es bastante común dentro del
reconocimiento de caracteres en prácticas de OCR \cite{hopfield1}. además en
clases se mencionaron otros usos de la red de Hopfield como para casos de
Optimizacion \cite{ia-hopfield}.

\subsection{Otras redes de aprendizaje no-supervisado}
Una de las características de las redes con aprendizaje no-supervisado, es su
capacidad de auto-organizarse, un ejemplo de estas ademas de las Redes de
Hopfield, son las redes de Mapas Autoorganizadas, tambien conocidas como Redes
Autooganizadas de Kohonen.

\subsection{Ejercicio de Hopfield}
El ejercicio para la red de hopfield se puede encontrar en el siguiente enlace:
\texttt{shorturl.at/hnq89}

% === Bilbiografia === %
\newpage
\printbibliography[title={Referencias}]
\end{document}
