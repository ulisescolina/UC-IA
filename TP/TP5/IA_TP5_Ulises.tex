\documentclass{article}
\renewcommand\refname{Referencias}
\renewcommand{\contentsname}{\'Indice de Conten\'ido}
\usepackage{graphicx}
\graphicspath{{img/}}
\usepackage{caption}
\usepackage{subcaption}
\usepackage{float}
\usepackage{amsmath}
\usepackage[spanish]{babel}
\usepackage[style=ieee]{biblatex}
\addbibresource{./IA_TP5_Ulises.bib}

\title{\textsc{Inteligencia Artificial y Sistemas Expertos}\\Trabajo Práctico
Número 5}
\author{\textsc{Ulises C. Ramirez} [\textit{ulir19@gmail.com}]}
\date{\today}
\begin{document}
\maketitle
\pagenumbering{gobble}
\newpage

\section*{Versionado}
Para el corriente documento se est\'a llevando un versionado a fin de mantener
un respaldo del trabajo y adem\'as proveer a la c\'atedra o a cualquier
interesado a la posibilidad de leer el material en la \'ultima versi\'on disponible.\\

\begin{center}
  \textsc{Repositorio}: \textit{https://github.com/ulisescolina/UC-IA}
\end{center}

\hfill--\textsc{Ulises}
\tableofcontents
\pagenumbering{gobble}
\newpage

% === Inicio del Cuerpo del Documento === %
\pagenumbering{arabic}
\textsc{Consigna}: \textit{
Definir las mejoras que incluyen los modelos Modelo de Síntomas Relevantes
Independientes y Modelo de Síntomas Relevantes Dependientes. Desarrollar un
ejemplo de cada uno. Responder ¿Cuál de ellos utiliza menor cantidad de
parámetros y porqué?}
\\

Como se ve en \cite{ia-sep}, la base del conocimiento dentro de un Sistema Experto
Probabilístico (SEP de ahora en más), consiste en un conjunto de variables y una o
varias funciones de probabilidad conjunta definida sobre ese conjunto de variables.

Se busca especificar directamente la funcion de probabilidad asignando un valor
o un parametro a cada combinacion de las variables mencionadas anteriormente.

Esto comienza a ser un inconveniente al tener un sistema que necesite muchas
variables para su modelización, para visualizar esto podemos tener un ejemplo
con variables binarias, es decir las diferentes variables que se tengan solo
pueden tomar 2 valores, para este escenario hipotético que se acaba de armar,
la funcion de probabilidad conjunta que se menciono tendra $2^{n}$ parámetros,
siendo $n$ la cantidad de variables binarias en el problema. Para un $n$
suficientemente grande se empiezan a ver los inconvenientes a afrontar.

En SEP, los problemas y las diferentes variables que lo modelen, pueden o no
estar relacionadas, es decir, puede o no existir dependencia entre ellas, este
factor nos permite reducir la complejidad de una estructura para modelar un
problema, que si bien es más simple en términos de interconexión de variables,
aún puede ser fiel al escenario.

De este razonamiento salen los Modelos de Sintomas Dependientes y los Modelos
de Sintomas Independientes.

\section{MSD}
El problema supone que existe dependencia entre los sintomas de las diferentes
enfermedades, pero que cada enfermedad es independiente de otra. Cada sintoma
puede relacionarse, no solamente con otros sintomas, sino que también con otras
enfermedades.

Para este modelo los parametros necesarios para la base de conocimiento serán:
\begin{itemize}
	\item Probabilidades marginales, para todos los valores posibles para $E$
		(enfermedades).
	\item Verosimilitudes de los sintomas, para cada enfermedad $P(s_{1}, s_{2},
		\hdots , s_{n} | e_{i})$, para todas las combinaciones posibles de síntomas y
		enfermedades.
\end{itemize}

Entonces, para $m$ enfermedades y $n$ sintomas binarios MSD requiere de un
total de $m \cdot 2^{n}-1$.

para valores de $m=2$ y $n=3$ se tendría:

\begin{equation}
	\begin{split}
		m \cdot 2^{n}-1 &= 2 \cdot 2^{3} - 1 \\
		&= 15
	\end{split}
\end{equation}

Ahora si se aumentan los valores para $m=100$ y $n=200$ se tiene que:
\begin{equation} \label{eq2}
	\begin{split}
		m \cdot 2^{n}-1 &= 100 \cdot 2^{200} - 1 \\
		&= 10^{62}
	\end{split}
\end{equation}

Esto hace al principal problema del MSD, el requerimiento de manejo de un
numero de parámetros muy alto.

\section{MSI}
Lo mencionado anteriormente, resulta menester la simplificación del modelo para
poder ser capaces de trabajar con un sistema que tenga una gran cantidad de
variables. Una de las simplificaciones a llevar a cabo es suponer que los
sintomas para una enfermedad dada son independientes entre si, esto da origen
al MSI.

Los parametros necesarios para la base de conocimiento del MSI son los
siguientes:
\begin{itemize}
	\item Probabilidades marginales, para todos los valores de $E$.
	\item Probabilidades condicionales $P(s_{j} | e_{i})$ para todos los sintomas
		y la enfermedad.
\end{itemize}

Se nota la reducción del numero de parametros necesarios para este acercamiento
del problema. si se tiene que la cantidad de enfermedades es $m$ y la cantidad
de síntomas binarios $n$, se requerirían de $m\cdot(n+1)-1$ parámetros.

para valores de $m=2$ y $n=3$ se tendría:

\begin{equation}
	\begin{split}
		m\cdot(n+1)-1 &= 2\cdot(3+1)-1 \\
		&= 7
	\end{split}
\end{equation}

Ahora si se aumentan los valores para $m=100$ y $n=200$ se tiene que:
\begin{equation}
	\begin{split}
		m\cdot(n+1)-1 &= 100\cdot(200+1)-1 \\
		&= 20099
	\end{split}
\end{equation}

Lo cual muestra una cantidad mucho menor que la obtenida en el Desarrollo
\ref{eq2}.
% === Bilbiografia === %
\newpage
\printbibliography[title={Referencias}]
\end{document}
